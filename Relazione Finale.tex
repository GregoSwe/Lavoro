\documentclass[11pt]{book}              % Book class in 11 points
\usepackage[utf8x]{inputenc} 
%\usepackage[italian]{babel} 
\parindent0pt  \parskip10pt             % make block paragraphs
\raggedright                            % do not right justify


%\usepackage{fancyhdr}

\usepackage{fancyhdr}
\pagestyle{fancy}
\evensidemargin=1cm 
\oddsidemargin=1cm


%\usepackage[latin1]{inputenc}

\usepackage{graphicx}
\usepackage{hyperref}
\hypersetup{
    colorlinks,
    citecolor=black,
    filecolor=black,
    linkcolor=black,
    urlcolor=black
}
\cfoot{}
\rfoot{\thepage\ di 11}
\renewcommand{\footrulewidth}{0.4pt}
\newcommand{\numref}[1]{\textsl{\nameref{#1} (\ref{#1})}}

\renewcommand{\chaptermark}[1]{%
\markboth{\thechapter.\ #1}{}}

\renewcommand{\sectionmark}[1]{\markright{\thesection.\ #1}}


\begin{document}
\frontmatter

\begin{titlepage}
\centering  
\textbf{\huge{Università degli Studi di Padova}} \\
\vspace{0.3cm}
\huge{Dipartimento di Matematica} \\
\vspace{0.3cm}
\LARGE{Corso di Laurea in Informatica} \\

\vspace{1cm}
\includegraphics[scale=0.2]{img/Logo_Unipd.png}  \\
\vspace{1cm}
\hspace{0.5cm}\textbf{\LARGE{ Sviluppo di dispositivo embedded per la gestione di un pluviometro con Arduino}}\\
\vspace{0.5cm}
\textit{Relazione Finale laurea triennale}\\
\vspace{0.5cm}

\begin{flushleft}
\textit{Relatore} \hfill \textit{Laureando}\\
Prof. Tullio Vardanega \hfill Marco Gregorini

\end{flushleft}
\line(1, 0){360}\\
\textsc{\small{Anno accademico 2015 - 2016}}
\end{titlepage}


\newpage
\thispagestyle{empty}
\vspace*{\fill}

Marco Gregorini: \textit{Sviluppo di dispositivo embedded per la gestione di un pluviometro con Arduino,} Relazione Finale laurea triennale, \copyright Feb 2015

\newpage




\chapter{Sommario}

[:] descrizione generale del documento e dei suoi capitoli.
                   

\newpage
\thispagestyle{empty}

\tableofcontents    

\newpage
\thispagestyle{empty}

                  
\mainmatter       
  

\chapter{Profilo dell'Azienda}

\thispagestyle{fancy} 

[:] Breve descrizione dell'azienda in generale, quando è stata fondata, quali sono le sue sedi operative e descrizione leggermente più dettagliata riguardo la sede dello stage.

\section{Cosa offre l'Azienda: Prodotti e Servizi}

[:] descrizione della sezione.

\subsection{I Prodotti di ARPAV}

[:] lista generale dei prodotti generati da ARPAV con breve descrizione.

\subsection{I Servizi di ARPAV}

[:] lista dei servizi offerti da ARPAV con breve descrizione.

\section{Strategie Aziendali}

[:] Descrizione della sezione con breve introduzione alle sottosezioni.

\subsection{Processi di Sviluppo}

[:] Descrizione dei processi attuati all'interno dell'azienda per lo sviluppo dei prodotti e la fruizione dei servizi e i metodi utilizzati per i processi di sviluppo.

\subsection{Metodologie di Supporto ai Processi}

[:] Elenco e descrizione delle metodologie utilizzate a supporto dei processi dall'azienda.

\subsection{Strumenti di Supporto ai Processi}

[:] Elenco e descrizione degli strumenti utilizzati dall'azienda a supporto dei processi.

\subsection{Tecnologie Utilizzate}

[:] Elenco e descrizione delle tecnologie utilizzate dall'azienda in relazione all'ambito in cui vengono applicate.

\section{Target Clienti}

[:] Descrizione dei clienti diretti dell'azienda e la metodologia di interazione; breve descrizione dei clienti secondari che usufruiscono dei prodotti erogati dall'azienda.

\subsection{Clienti Diretti}

[:]  descrizione particolare dei clienti diretti dei servizi e prodotti dell'azienda e le metodologie di relazione con essi

\subsection{Clienti Secondari}

[:] descrizione dei clienti secondari, ma spesso finali dei prodotti e servizi forniti dall'azienda.


\chapter{Lo Stage per Azienda}
\thispagestyle{fancy} 

[:] introduzione del capitolo con la descrizione dei motivi per cui l'azienda ricerca stagisti per affidare loro dei progetti da sviluppare.

\section{Presentazione del Progetto}

[:] Sezione in cui descrivo il progetto vero e proprio, in che cosa consiste e in quali realtà potrà essere inserito.

\subsection{Obbiettivo dello Stage}

[:] Sottosezione in descrivo gli obbiettivi che sono stati preposti all'inizio dello stage nel momento in cui ho preso visione dell'offerta fatta dall'azienda.
La sezione inizia con una breve introduzione del piano di lavoro e la descrizione del significato di obbiettivo.

\subsubsection{Obbiettivi Minimi Richiesti}

[:] Elenco e descrizione degli obbiettivi minimi richiesti all'inizio dello stage.

\subsubsection{Obbiettivi Massimi Raggiungibili}

[:] Elenco e descrizione degli obbiettivi che l'azienda desidera raggiungere, ma che non sono vincolanti alla conclusione dello stage.

\subsection{Finalità del Progetto}

[:] Sezione descrivo in dettaglio l'applicazione del prodotto dello stage nella realtà aziendale e non.

\section{Vincoli di Progetto}

[:] Descrizione delle tipologie di vincoli che sono stati imposti nello stage.

\subsection{Vincoli di Dominio}

[:] Descrizione dei vincoli che dovranno essere soddisfatti dal progetto in termini di funzionalità, affidabilità e manutenibilità. 

\subsection{Vincoli Tecnologici}

[:] Descrizione dei vincoli riguardanti gli strumenti di lavoro, come framework e  e tecnologie obbligatori che dovranno essere utilizzati durante l'attività stagistica.

\subsection{Vincoli Metodologici}

[:] Descrizione dei vincoli metodologici che lo ho dovuto seguire durante tutta la fase dello stage.

\subsection{Vincoli Temporali}

[:] Descrizione dei vincoli temporali prefissati in base alle necessità aziendali e i problemi di fattori esterni dovuti alla cooperazione di aziende terze per il completamento dello stage.

\chapter{L'Attività di Stage}
\thispagestyle{fancy} 

[:] Descrizione introduttiva del capitolo e delle sue sezioni

\section{Pianificazione del Lavoro}

[:] Sezione in cui descrivo la pianificazione a monte di tutta l'attività di stage in relazione con i vincoli precedentemente assunti assieme all'azienda.

\subsection{Problematiche Riscontrate}

[:] Sottosezione in cui descrivo le problematiche che si sono riscontrate durante lo stage che hanno fatto deviare il piano di lavoro precedentemente da me stabilito in un altro.

\section{Preparazione al Lavoro di Stage}

[:] Sezione in cui descrivo come mi sono preparato per affrontare il lavoro dello stage il meno impreparato possibile, così da poter attutire possibili ritardi.

\subsection{Studio del Dominio}

[:] Sottosezione in cui spiego come mi sono affacciato al dominio richiesto per lo stage, assumendone tutti i pro, i contro, possibilità e criticità che ho riscontrato all'inizio dell'attività di stage.

\subsection{Studio delle Tecnologie}

[:] Sottosezione in cui descrivo, uno per uno, le tecnologie e gli strumenti necessarie per intraprendere lo stage e i supporti utilizzati per acquisire tali conoscenze.

\section{L'Utilizzo dei Prototipi}

[:] Breve sezione in cui spiego come, a causa delle necessità aziendali, mi sia trovato costretto ad un continuo ciclo di "Progettazione, Codifica, Test, Progettazione Incrementale, Codifica, Test..", approccio diverso da quello affrontato durante il mio percorso accademico.

\section{Attività di Analisi}
[:] Introduzione alla sezione e breve descrizione delle sottosezioni


\subsection{Classificazione dei Requisiti}

[:] Descrizione delle suddivisioni fatte per classificare i requisiti in modo tale da risultare chiari e comprensibili.


\subsection{Individuazione dei Requisiti}

[:] Descrizione dei metodi utilizzati per individuare i requisiti e visualizzazione degli stessi tramite tabella

\subsubsection{Strumenti Utilizzati}

[:] Elenco e descrizione degli strumenti utilizzati per il tracciamento dei requisiti

\subsection{Casi d'Uso}

[:] Descrizione del motivo per cui i casi d'uso non sono stati usati come mi ero prefissato ad inizio stage.

\subsection{Dettagli Meritevoli}

[:] Sezione in cui prendo in considerazione alcuni requisiti particolari analizzandone gli aspetti di difficoltà, problematiche e metodi di risoluzione del problema.

\section{Progettazione Architetturale}

[:] Descrizione dell'attività di 


\end{document}