\documentclass[11pt]{book}              % Book class in 11 points
%\usepackage[utf8x]{inputenc} 
%\usepackage[italian]{babel} 
\parindent0pt  \parskip10pt             % make block paragraphs
\raggedright                            % do not right justify


%\usepackage{fancyhdr}

\usepackage{fancyhdr}
\pagestyle{fancy}
\evensidemargin=1cm 
\oddsidemargin=1cm


%\usepackage[latin1]{inputenc}

\usepackage{graphicx}
\usepackage{hyperref}
\hypersetup{
    colorlinks,
    citecolor=black,
    filecolor=black,
    linkcolor=black,
    urlcolor=black
}
\cfoot{}
\rfoot{\thepage\ di 2}
\renewcommand{\footrulewidth}{0.4pt}
\newcommand{\numref}[1]{\textsl{\nameref{#1} (\ref{#1})}}

\begin{document}
\frontmatter

\begin{titlepage}
\centering  
\textbf{\huge{Università degli Studi di Padova}} \\
\vspace{0.3cm}
\huge{Dipartimento di Matematica} \\
\vspace{0.3cm}
\LARGE{Corso di Laurea in Informatica} \\

\vspace{1cm}
\includegraphics[scale=0.2]{img/Logo_Unipd.png}  \\
\vspace{1cm}
\hspace{0.5cm}\textbf{\LARGE{ Sviluppo di dispositivo embedded per la gestione di un pluviometro con Arduino}}\\
\vspace{0.5cm}
\textit{Relazione Finale laurea triennale}\\
\vspace{0.5cm}

\begin{flushleft}
\textit{Relatore} \hfill \textit{Laureando}\\
Prof. Tullio Vardanega \hfill Marco Gregorini

\end{flushleft}
\line(1, 0){360}\\
\textsc{\small{Anno accademico 2015 - 2016}}
\end{titlepage}


\newpage
\thispagestyle{empty}
\vspace*{\fill}

Marco Gregorini: \textit{Sviluppo di dispositivo embedded per la gestione di un pluviometro con Arduino,} Relazione Finale laurea triennale, \copyright Feb 2015

\newpage

\chapter{Sommario}
[:] descrizione generale del documento e dei suoi capitoli
                          

\newpage
\thispagestyle{empty}

\tableofcontents    

\newpage
\thispagestyle{empty}

                  
\mainmatter       
  

\chapter{Profilo dell'Azienda}
\flushleft
\thispagestyle{fancy} 

[:] Breve descrizione dell'azienda in generale, quando è stata fondata, quali sono le sue sedi operative e descrizione leggermente più dettagliata riguardo la sede dello stage.

\section{Cosa offre l'Azienda: Prodotti e Servizi}

[:] descrizione della sezione

\subsection{I Prodotti di ARPAV}

[:] lista generale dei prodotti generati da ARPAV con breve descrizione

\subsection{I Servizi di ARPAV}

[:] lista dei servizi offerti da ARPAV con breve descrizione

\section{Strategie Aziendali}

[:] Descrizione della sezione con breve introduzione alle sottosezioni

\subsection{Processi di Sviluppo}

[:] Descrizione dei processi attuati all'interno dell'azienda per lo sviluppo dei prodotti e la fruizione dei servizi e i metodi utilizzati per i processi di sviluppo

\subsection{Strumenti di supporto ai Processi di Sviluppo}

[:] Elenco e descrizione degli strumenti utilizzati dall'azienda a supporto dei processi.

\subsection{Tecnologie utilizzate}

[:] Elenco e descrizione delle tecnologie utilizzate dall'azienda in relazione all'ambito in cui vengono applicate

\section{Clientela Target}

[:] Descrizione dei clienti diretti dell'azienda e la metodologia di interazione; breve descrizione dei clienti secondari che usufruiscono dei prodotti erogati dall'azienda.



\end{document}