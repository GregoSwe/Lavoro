\documentclass[11pt]{book}              % Book class in 11 points
\usepackage[utf8x]{inputenc} 
%\usepackage[italian]{babel} 
\parindent0pt  \parskip10pt             % make block paragraphs
\raggedright                            % do not right justify


%\usepackage{fancyhdr}

\usepackage{fancyhdr}
\pagestyle{fancy}
\evensidemargin=1cm 
\oddsidemargin=1cm


%\usepackage[latin1]{inputenc}

\usepackage{graphicx}
\usepackage{hyperref}
\hypersetup{
    colorlinks,
    citecolor=black,
    filecolor=black,
    linkcolor=black,
    urlcolor=black
}
\cfoot{}
\rfoot{\thepage\ di 4}
\renewcommand{\footrulewidth}{0.4pt}
\newcommand{\numref}[1]{\textsl{\nameref{#1} (\ref{#1})}}

\renewcommand{\chaptermark}[1]{%
\markboth{\thechapter.\ #1}{}}

\renewcommand{\sectionmark}[1]{\markright{\thesection.\ #1}}


\begin{document}
\frontmatter

\begin{titlepage}
\centering  
\textbf{\huge{Università degli Studi di Padova}} \\
\vspace{0.3cm}
\huge{Dipartimento di Matematica} \\
\vspace{0.3cm}
\LARGE{Corso di Laurea in Informatica} \\

\vspace{1cm}
\includegraphics[scale=0.2]{img/Logo_Unipd.png}  \\
\vspace{1cm}
\hspace{0.5cm}\textbf{\LARGE{ Sviluppo di dispositivo embedded per la gestione di un pluviometro con Arduino}}\\
\vspace{0.5cm}
\textit{Relazione Finale laurea triennale}\\
\vspace{0.5cm}

\begin{flushleft}
\textit{Relatore} \hfill \textit{Laureando}\\
Prof. Tullio Vardanega \hfill Marco Gregorini

\end{flushleft}
\line(1, 0){360}\\
\textsc{\small{Anno accademico 2015 - 2016}}
\end{titlepage}


\newpage
\thispagestyle{empty}
\vspace*{\fill}

Marco Gregorini: \textit{Sviluppo di dispositivo embedded per la gestione di un pluviometro con Arduino,} Relazione Finale laurea triennale, \copyright Feb 2015

\newpage

\chapter{Sommario}
[:] descrizione generale del documento e dei suoi capitoli.
                          

\newpage
\thispagestyle{empty}

\tableofcontents    

\newpage
\thispagestyle{empty}

                  
\mainmatter       
  

\chapter{Profilo dell'Azienda}

\thispagestyle{fancy} 

[:] Breve descrizione dell'azienda in generale, quando è stata fondata, quali sono le sue sedi operative e descrizione leggermente più dettagliata riguardo la sede dello stage.

\section{Cosa offre l'Azienda: Prodotti e Servizi}

[:] descrizione della sezione.

\subsection{I Prodotti di ARPAV}

[:] lista generale dei prodotti generati da ARPAV con breve descrizione.

\subsection{I Servizi di ARPAV}

[:] lista dei servizi offerti da ARPAV con breve descrizione.

\section{Strategie Aziendali}

[:] Descrizione della sezione con breve introduzione alle sottosezioni.

\subsection{Processi di Sviluppo}

[:] Descrizione dei processi attuati all'interno dell'azienda per lo sviluppo dei prodotti e la fruizione dei servizi e i metodi utilizzati per i processi di sviluppo.

\subsection{Strumenti di Supporto ai Processi di Sviluppo}

[:] Elenco e descrizione degli strumenti utilizzati dall'azienda a supporto dei processi.

\subsection{Tecnologie Utilizzate}

[:] Elenco e descrizione delle tecnologie utilizzate dall'azienda in relazione all'ambito in cui vengono applicate.

\section{Target Clienti}

[:] Descrizione dei clienti diretti dell'azienda e la metodologia di interazione; breve descrizione dei clienti secondari che usufruiscono dei prodotti erogati dall'azienda.

\subsection{Clienti Diretti}

[:] in questa sottosezione verranno descritti in particolare i clienti diretti dei servizi e prodotti dell'azienda e le metodologie di relazione con essi

\subsection{Clienti Secondari}

[:] in questa sottosezione verranno descritti invece i clienti secondari, ma spesso finali dei prodotti e servizi forniti dall'azienda.

\newpage
\chapter{Lo Stage Visto dall'Azienda}
\thispagestyle{fancy} 
[:] introduzione del capitolo con la descrizione dei motivi per cui l'azienda ricerca stagisti per affidare loro dei progetti da sviluppare.

\section{Presentazione del Progetto}

[:] Sezione in cui viene descritto il progetto vero e proprio, in cosa consiste e in quali realtà potrà essere inserito.

\subsection{Obbiettivo dello Stage}

[:] Sottosezione in cui si descrivono gli obbiettivi che sono stati preposti all'inizio dello stage nel momento in cui ho preso visione dell'offerta fatta dall'azienda.
La sezione inizia con una breve introduzione del piano di lavoro e la descrizione del significato di obbiettivo.

\subsubsection{Obbiettivi Minimi Richiesti}

[:] Vengono elencati e descritti gli obbiettivi minimi richiesti all'inizio dello stage.

\subsubsection{Obbiettivi Massimi Raggiungibili}

[:] Vengono elencati e descritti gli obbiettivi che l'azienda desidera raggiungere, ma che non sono vincolanti alla conclusione dello stage.

\section{Vincoli di Progetto}

[:] La sezione inizia descrivendo le tipologie di vincoli che sono stati imposti nello stage

\subsection{Vincoli di Dominio}

[:] Vengono descritti i vincoli che dovranno essere soddisfatti dal progetto in termini di funzionalità, affidabilità e manutenibilità. 

\subsection{Vincoli Tecnologici}

[:] Vengono descritti i vincoli riguardanti gli strumenti di lavoro, come framework e  e tecnologie obbligatori che dovranno essere utilizzati durante l'attività stagistica.


\subsection{•}



\end{document}