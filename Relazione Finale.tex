\documentclass[11pt]{book}              % Book class in 11 points
\usepackage[utf8x]{inputenc} 
\usepackage[italian]{babel} 
\parindent0pt  \parskip10pt             % make block paragraphs
\raggedright                            % do not right justify

\usepackage{longtable,array,booktabs,graphicx}
\usepackage{color}
\usepackage[usenames,dvipsnames]{xcolor}
\usepackage{multirow}

\usepackage{amsmath}
\usepackage{graphicx}
%\usepackage{fancyhdr}
\usepackage{caption}
\usepackage{longtable}

\usepackage{fancyhdr}
\pagestyle{fancy}
\evensidemargin=1cm 
\oddsidemargin=1cm


%\usepackage[latin1]{inputenc}

\usepackage{graphicx}
\usepackage{hyperref}
\hypersetup{
    colorlinks,
    citecolor=black,
    filecolor=black,
    linkcolor=black,
    urlcolor=blue
}


\renewcommand{\chaptermark}[1]{
.\markboth{\thechapter.\ #1}{}}

\renewcommand{\sectionmark}[1]{\markright{\thesection.\ #1}}

\headsep=0pt
\topmargin=0pt
\headheight=0.7in
\headsep=25pt
\textheight=550pt
\textwidth=444pt
\marginparsep=0pt
\marginparwidth=0pt
\footskip=2cm
\hoffset=0pt
\voffset=0pt



%%%%%%%%%%%%%
\begin{document}
\frontmatter

\begin{titlepage}
\centering  
\textbf{\huge{Università degli Studi di Padova}} \\
\vspace{0.3cm}
\huge{Dipartimento di Matematica} \\
\vspace{0.3cm}
\LARGE{Corso di Laurea in Informatica} \\

\vspace{1cm}
\includegraphics[scale=0.2]{img/Logo_Unipd.png}  \\
\vspace{1cm}
\hspace{0.5cm}\textbf{\LARGE{ Sviluppo di dispositivo embedded per la gestione di un pluviometro con Arduino}}\\
\vspace{0.5cm}
\textit{Relazione Finale laurea triennale}\\
\vspace{0.5cm}

\begin{flushleft}
\textit{Relatore} \hfill \textit{Laureando}\\
Prof. Tullio Vardanega \hfill Marco Gregorini

\end{flushleft}
\line(1, 0){360}\\
\textsc{\small{Anno accademico 2015 - 2016}}
\end{titlepage}


\newpage
\thispagestyle{empty}
\vspace*{\fill}

Marco Gregorini: \textit{Sviluppo di dispositivo embedded per la gestione di un pluviometro con Arduino,} Relazione Finale laurea triennale, \copyright Feb 2015

\newpage




\chapter*{Sommario}

\section*{Premessa}


Il seguente documento ha lo scopo di illustrare il lavoro svolto durante l'attività di stage dallo laureando Gregorini Marco, presso la sede del servizio informatico e di reti dell'agenzia regionale ARPAV, a Padova. Lo studente intende descrivere in modo critico e oggettivo le attività svolte durante il suo percorso di lavoro, della durata di circa trecentoventi ore, e i riscontri avuti con questa esperienza.

\section*{Contenuti}

Nel documento vengono descritte le attività svolte per il completamento del progetto proposto da ARPAV, le necessità che hanno creato l'opportunità di lavoro per uno studente laureando e le prospettive che il progetto apre per il futuro.
I contenuti vengono esposti in quattro capitoli:

\begin{enumerate}
	\item \textbf{\nameref{1.0}:} descrizione generale dell'azienda. Dall'organizzazione interna agli obbiettivi che si pone;
	\item \textbf{\nameref{2.0}:} lo stage visto dal punto di vista dell'ente. Dai motivi che hanno evidenziato una necessità di uno stage, alla presentazione del progetto;
	\item \textbf{\nameref{3.0}:} lo stage visto dal punto di vista dello studente. Dalla pianificazione alla realizzazione;
	\item \textbf{\nameref{4.0}:} analisi retrospettiva dell'esperienza fatta. Valutazione critica dei risultati ottenuti, descrizione delle capacità e abilità acquisite, valutazione dello svolgimento del lavoro svolto.
\end{enumerate}

Nel documento sono state introdotto delle convenzioni topografiche per agevolare la lettura e la comprensione dei contenuti:

\begin{itemize}
	\item \textbf{Glossario :}  i termini che richiedono una definizione più accurata, di valore tecnico o di uso non comune, verranno descritti all'interno del \nameref{Glossario} e evidenziati nel testo con una "g" a pedice alla loro prima occorrenza: parola\ped{g};
	\item \textbf{Termini Inglesi:} i termini in lingua inglese verranno evidenziati in corsivo: \textit{parola};


\end{itemize}
\newpage
\thispagestyle{empty}

\tableofcontents    

\listoffigures

\listoftables

\newpage
\thispagestyle{empty}

\cfoot{}
\rfoot{\thepage\ di 28}
\renewcommand{\footrulewidth}{0.2pt}
\newcommand{\numref}[1]{\textsl{\nameref{#1} (\ref{#1})}}                
\mainmatter       
  


\chapter{Profilo dell'Agenzia}
\label{1.0}
\thispagestyle{fancy} 

[:] Breve descrizione dell'agenzia in generale, quando è stata fondata, quali sono le sue sedi operative e descrizione leggermente più dettagliata riguardo la sede dello stage.

\section{Cosa offre: Prodotti e Servizi}

[:] descrizione della sezione.

\subsection{I Prodotti di ARPAV}

[:] lista generale dei prodotti generati da ARPAV con breve descrizione.

\subsection{I Servizi di ARPAV}

[:] lista dei servizi offerti da ARPAV con breve descrizione.

\section{Organizzazione Interna}

[:] Descrizione della sezione con breve introduzione alle sottosezioni.

\subsection{Processi di Sviluppo}

[:] Descrizione dei processi attuati all'interno dell'agenzia per lo sviluppo dei prodotti e la fruizione dei servizi e i metodi utilizzati per i processi di sviluppo.

\subsection{Metodologie di Supporto ai Processi}

[:] Elenco e descrizione delle metodologie utilizzate a supporto dei processi dall'agenzia.

\subsection{Strumenti di Supporto ai Processi}

[:] Elenco e descrizione degli strumenti utilizzati dall'agenzia a supporto dei processi.

\subsection{Tecnologie Utilizzate}

[:] Elenco e descrizione delle tecnologie utilizzate dall'agenzia in relazione all'ambito in cui vengono applicate.

\section{Relazioni Esterne}

[:] Descrizione dei clienti diretti dell'agenzia e la metodologia di interazione; breve descrizione dei clienti secondari che usufruiscono dei prodotti erogati dall'agenzia.

\subsection{Clienti Diretti}

[:]  descrizione particolare dei clienti diretti dei servizi e prodotti dell'agenzia e le metodologie di relazione con essi

\subsection{Clienti Secondari}

[:] descrizione dei clienti secondari, ma spesso finali dei prodotti e servizi forniti dall'agenzia.

\subsection{Orientamento all'Innovazione}

[:] Descrizione della propensione dell'ente regionale all'innovazione in relazione con i suoi prodotti e servizi e progetti futuri.



\chapter{Lo Stage per ARPAV}
\label{2.0}
\thispagestyle{fancy} 

Il dipartimento di reti ed informatica deve affrontare la difficile sfida di mantenere la parità del bilancio, senza rinunciare alla qualità delle loro opere. Il responsabile del dipartimento, spinto da questa necessità, ha orientato la politica di scelta del personale, durante la fase di codifica, verso stagisti neo-laureati o laureandi. Questa scelta permette:

\begin{itemize}

	\item \textbf{maggior efficacia:} l'utilizzo di stagisti permette di ottimizzare le risorse di bilancio, diminuendo l'impatto sul \textit{budget};
	\item \textbf{maggiore efficacia:} menti fresche di studio e con bassa specializzazione permettono un'adesione a progetti nuovi ed innovativi con maggiore facilità rispetto a personale propenso a lavorare secondo processi prestabiliti.

\end{itemize}

\section{L'Esigenza di uno Stage}

Il dipartimento di reti ed informatica, al momento del mio inserimento, stava seguendo la progettazione e lo sviluppo di un pluviometro\ped{g} innovativo a basso costo da parte del \textit{FabLab}\ped{g} di Verona.
\begin{figure}[htbp]
\centering
\begin{minipage}[c]{.30\textwidth}
	\centering\setlength{\captionmargin}{0pt}%
	\includegraphics[width=0.50\textwidth]{./capitoli/capitolo2/img/raspibo}
	\caption{Logo RaspiBO}
\end{minipage}%
\hspace{10mm}
\begin{minipage}[c]{.30\textwidth}
	\centering\setlength{\captionmargin}{0pt}%
	\includegraphics[width=0.90\textwidth]{./capitoli/capitolo2/img/fablab}
	\caption{Logo FabLab}
\end{minipage}%
\end{figure} Questo incarico è stato proposto da parte di ARPAV per il progetto RE.S.M.I.A., il quale prevede il potenziamento dell'infrastruttura della rete di monitoraggio di ARPAV, tramite la progettazione di nuove stazioni meteorologiche. 
Contemporaneamente al progetto \textit{FabLab}, il responsabile del dipartimento stava seguendo il \textit{Gruppo Meteo} di \textit{RaspiBO}\ped{g} il quale sta sviluppando stazioni meteorologiche a basso costo, al momento sprovviste di pluviometro. La combinazione di questi due fattori ha creato la necessità di sviluppo di un sistema \textit{hardware} e \textit{software} che permettesse la gestione di un pluviometro tramite tecnologie a basso costo, ma con prestazioni adeguate alle richieste.



\section{Presentazione del Progetto}

La \textit{Small Project}, azienda italiana nata del 2004, appoggia il progetto \textit{Arduino} e ne è diventata il produttore principale. Arduino è una scheda elettronica di piccole dimensioni, con un microcontrollore e circuiteria di contorno per la realizzazione di progetti o prototipi. Il progetto Arduino, ormai avviato da qualche anno, ora conta svariate tipologie di schede con caratteristiche di prestazioni e prezzo differenti fra loro. Queste schede permettono una facile programmazione e una elastica implementazione a seconda delle necessità che il proprio lavoro richiedono. Per il progetto di \textit{stage}, è stato proposto l'utilizzo di una scheda \textit{Arduino UNO} (\url{www.arduino.cc/en/Main/ArduinoBoardUno}).

\begin{figure}[htbp]
\centering
\begin{minipage}[c]{.60\textwidth}
	\centering\setlength{\captionmargin}{0pt}
	\includegraphics[scale=0.5]{./capitoli/capitolo2/img/ArduinoUNO}
	\caption{Scheda Arduino UNO}
\end{minipage}
\hspace{10mm}
\begin{minipage}[b]{.30\textwidth}
	\centering\setlength{\captionmargin}{0pt}%
	\includegraphics[width=0.30\textwidth]{./capitoli/capitolo2/img/arduinologo}
	\caption{Logo Arduino}
\end{minipage}%

\end{figure}  

Il progetto prevede la progettazione e lo sviluppo di un dispositivo innovativo \textit{embedded} per la gestione di un pluviometro, lo studio per una gestione e memorizzazione efficiente di eventi di pioggia
mediante standardizzazione su protocollo MQTT\ped{g}, sviluppato su \textit{hardware} e \textit{firmware} ATMEL\ped{g}.

\begin{figure}[htbp]
	\centering
	\includegraphics[scale=1]{./capitoli/capitolo2/img/mqtt}
	\caption{Esempio protocollo MQTT}

\end{figure}

Le componenti hardware saranno volutamente poco costose e quindi con limitate capacità di memorizzazione dati, pertanto sarà necessaria la ricerca di soluzioni che tengano in considerazione tale problematica. Nel progetto verrà richiesto l'utilizzo di due schede Arduino UNO, di cui una con lo scopo di gestione dei segnali ricevuti dall'\textit{hardware} del pluviometro, salvataggio e gestione dei dati, l'altra con la funzionalità di interrogazione della prima. Si sono riscontrate così due "tipologie" di schede:
\begin{itemize}
	\item \textbf{Scheda Slave:} scheda che si interfaccia direttamente con i segnali inviati dal pluviometro, che salva all'interno di una memoria i dati richiesti e li cancella in caso di necessità o richiesta;
	\item \textbf{Scheda Master:} scheda che permette l'interrogazione della \textit{scheda Slave}, tramite protocollo prestabilito e rimbalzi così i dati ricevuti all'utente che ne ha fatto richiesta.

\end{itemize}

La \textit{scheda Slave}, dovrà essere interamente progettata e realizzata all'interno del lavoro di \textit{stage}, per quanto riguarda la \textit{scheda master}, ciò che viene richiesto è la creazione di una libreria che permetta un facile accesso alle funzionalità di base richieste dal progetto e un'architettura tale che permetta facilmente integrazioni future.
Viene richiesto in oltre che tutto  progetto di \textit{stage} venga rilasciato tramite \textit{GNU GPL}\ped{g} su \textit{github}, provvisto di documentazione dei metodi e componenti \textit{software} associata per facilitare future modifiche o correzioni.


\subsection{Obiettivo dello Stage}

Gli obiettivi sono stati preposti durante i colloqui conoscitivi precedenti all'inizio dello \textit{stage}, evidenziando quelle che sarebbero state le caratteristiche salienti sufficienti per considerare soddisfacente il lavoro completo del tirocinio.

\begin{itemize}

	\item \textbf{Obiettivi esplorativi:} evidenziano il livello di capacità di indagine dello stagista nell'affrontare un ambiente nuovo in cui vengono utilizzate tecnologie mai affrontate durante il percorso accademico. Viene anche delineata l'attitudine ad analizzare i rischi legati all'utilizzo di componenti \textit{hardware} inesplorati da parte dell'agenzia e la capacità di risoluzioni dei problemi legati alle limitate prestazioni delle schede Arduino;
	
	\item \textbf{Obiettivi funzionali:} includono i requisiti minimi richiesti delle funzionalità per considerare il lavoro di \textit{stage} sufficiente alla propria conclusione. I requisiti descrivono le funzionalità che dovranno essere implementate nel progetto e rappresentano l'intento centrale di tutto il lavoro del tirocinio.
	
	\item \textbf{Obiettivi funzionali opzionali:} rappresentano l'insieme di requisiti che descrivono le funzionalità del progetto che non sono obbligatori per la completezza del lavoro di tirocinio, ma che rappresentano un valore aggiunto e desiderato da parte dei proponenti.
\end{itemize}

\subsubsection{Obiettivi Minimi Richiesti}
Di seguito vengono riportati gli obiettivi minimi che sono stati richiesti al momento di inizio \textit{stage}:

\begin{itemize}

	\item \textbf{Obiettivi esplorativi:}
	\begin{itemize}
	
		\item \textit{Acquisizione conoscenze elettronica di base:} dovendo affrontare un lavoro con schede \textit{hardware}, connettendo cavi, creando connessioni con dispositivi ausiliari e correnti elettriche, è strettamente necessario acquisire le conoscenze di base che permettano di lavorare in sicurezza e senza causare guasti alle strumentazioni;
		\item \textit{Comprensione ambiente Arduino:} il mondo Arduino offre una vasta scelta di modi in cui operare, dalla tipologia di linguaggio, agli strumenti di sviluppo. 
		\item \textit{Valutazione prestazione Arduino:} uno dei punti focali stabiliti ad inizio \textit{stage} è stato comprendere il prima possibile se la scheda fornita fosse sufficientemente adeguata per le richieste del progetto. Pur non essendo un vero e proprio obiettivo, mi è stato preposto come tale a causa della sua importanza, poiché la ricerca di un nuovo componente, o anche la sola acquisizione \textit{online}, avrebbe provocato considerevoli ritardi.
	
	\end{itemize}
	\item \textbf{Obiettivi funzionali:}
		\begin{itemize}
			\item \textit{Progettazione e sviluppo scheda Slave:} punto focale del lavoro di \textit{stage} è lo sviluppo della scheda connessa al pluviometro, che dovrà:
			\begin{itemize}
				\item Riconoscere gli impulsi inviati dal prototipo di pluviometro;
				\item Riconoscere segnali falsi positivi e non registrare dati infondati;
				\item Immagazzinare in una memoria i \textit{timestamp};
				\item Ottimizzare la gestione della memoria;
				\item Comunicare tramite \textit{protocollo I2C}\ped{g} con le componenti a lei connesse;
				\item Comunicare tramite protocollo I2C con la componente Master a due vie;
				\item Ritorno dei dati memorizzati;
				\item Cancellazione dei dati;
				\item Tenere traccia di eventuali \textit{overflow} dovuti alle basse prestazioni di memoria;
				\item Progettazione elastica che permetta l'inserimento futuro di nuovi protocolli di comunicazione o inserimento di nuovi componenti hardware;

			\end{itemize}
			\item \textit{Progettazione e sviluppo libreria Master:} secondo punto focale del progetto, sviluppo di una libreria che possa essere inserita in un contesto dove è già esistente una scheda Master a cui viene connesso il pluviometro con scheda Arduino integrata con il sistema Slave:
			\begin{itemize}
				\item Comunicazione tramite protocollo I2C con la scheda Slave del pluviometro;
				\item Progettazione elastica che permetta l'inserimento futuro di nuovi protocolli per la comunicazione con la scheda Slave;
				\item Richiesta per la ricezione dei dati inseriti in memoria della scheda Slave;
				\item Richiesta per la ricezione dei dati ancora non letti;
				\item Richiesta per la cancellazione dei dati già letti;
				\item Richiesta per la cancellazione dell'intera memoria della scheda Slave;
			\end{itemize}
		\end{itemize}

\end{itemize}



\subsubsection{Obiettivi Massimi Raggiungibili}

Di seguito vengono riportati gli obiettivi preposti ad inizio \textit{stage}, che non sono vincolanti alla conclusione dello stesso:
\begin{itemize}
		\item \textbf{Obiettivi esplorativi:} 
		\begin{itemize}
			\item \textit{Acquisizione conoscenze elettronica semi-avanzate:} imparare a formare uno schema elettronico di una scheda in modo da realizzare una componente \textit{hardware} nuova partendo da componenti separate e l'utilizzo di strumenti che ne permettano l'assemblo.
		\end{itemize}

		\item \textbf{Obiettivi funzionali opzionali:}
		\begin{itemize}
			\item \textit{Implementazione del codice su scheda Arietta\ped{g} :} adattamento, implementazione e test del codice precedentemente approvato su Arduino UNO in una scheda Arietta;
			\item \textit{Interfaccia di configurazione e ricezioni dati web o desktop:} progettazione e realizzazione di un \textit{tool web} o \textit{desktop} per la relazione diretta \textit{computer/web}-scheda Master;
			\item \textit{Comunicazione 3G con scheda Master:} implementazione di un dispositivo per la comunicazione via 3G tra dispositivo \textit{mobile} e la scheda Master;
		\end{itemize}

\end{itemize}
\subsection{Finalità del Progetto}
\begin{figure}[htbp]
\centering
\includegraphics[scale=.3]{./capitoli/capitolo2/img/pluviometro}
\caption{Prototipo pluviometro FabLab}
\end{figure}

Nella realtà ARPAV il suddetto lavoro di \textit{stage}, oltre ad integrare una parte mancante del progetto RE.S.M.I.A., trova largo margine di prospettive future. Il progetto verrà rilasciato con licenza di utilizzo GNU GPL, quindi liberamente utilizzabile da chiunque. FabLab di Verona ha già lanciato un progetto parallelo successivo al lavoro di \textit{stage}: \textit{Tutti misurano, OpenPluvio}. Il progetto riprende il lavoro portato a termine e lo continua per introdurre un sistema di misurazione ambientale casalingo, amatoriale, che possa inviare i dati delle misurazioni effettuate a \textit{serve} dedicati. Lo scopo è facilitare il possesso di strumentazione di controllo ambientale a basso costo e la creazione una nuova rete di monitoraggio che possa integrare o arricchire i dati delle reti ufficiali.



\section{Vincoli di Progetto}

In questa sezione verranno descritti i vincoli del progetto che sono stati stabiliti ad inizio \textit{stage}, catalogati nel seguente modo:
\begin{itemize}
\item \textbf{Vincoli di dominio:} descrivono i vincoli di funzionalità, affidabilità e manutenibilità;
\item \textbf{Vincoli tecnologici:} descrivono gli strumenti di lavoro e le tecnologie obbligatorie;
\item \textbf{Vincoli metodologici:} descrivono le metodologie che imposte durante fase di \textit{stage};
\item \textbf{Vincoli temporali:} descrivono tappe obbligatorie da seguire durante l'attività lavorativa;
\end{itemize}

\subsection{Vincoli di Dominio}

Il sistema da realizzare dovrà immagazzinare una discreta quantità di dati in uno spazio di memoria ristretto. La necessità di sviluppare un algoritmo di compressione dei dati sarà estremamente necessario. I dati che dovranno essere immagazzinati vengono definiti \textit{timestamp}. I \textit{timestamp} sono un \textit{metadato} composto da più elementi:
\begin{itemize}
	\item \textbf{Data:} nel formato gg / mm / aaaa (esempio  12 05 2014);
	\item \textbf{Ora:} nel formato hh / mm / ss (esempio 12 02 35);
	\item \textbf{Contatore:} un numero che deve identificare il numero di \textit{timestamp} dall'ultima lettura effettuata.	
\end{itemize}

Ogni \textit{timestamp} verrà salvato in memoria successivamente all'interruzione di una basculata da parte del pluviometro. Una basculata avviene quando la vaschetta della bascula viene riempita fino ad un livello tale che il peso dell'acqua fa cadere la vaschetta. A seconda della tipologia di bascula il segnale può essere inviato diversamente. 

\begin{figure}[htpb]
\centering
\includegraphics[scale=.4]{./capitoli/capitolo2/img/bascula}
\caption{Bascula di un pluviometro}

\end{figure}


Durante il lavoro di \textit{stage} verrà usato un pluviometro ancora sperimentale, quindi lo strumento di misurazione della bascula potrebbe risultare instabile e produrre dei segnali falsi positivi, con il risultato di immagazzinare erroneamente dei \textit{timestamp}. Uno dei vincoli principali del sistema dovrà essere l'affidabilità dei dati, i quali dovranno essere affidabili al 100\%. 

Il codice prodotto verrà rilasciato con licenza d'uso GPL, questo richiede che il codice sia chiaro e mantenibile. Per questo motivo è richiesta la presenza di documentazione che descriva le componenti \textit{software} e i metodi che contengono.

\subsection{Vincoli Tecnologici}

In questa sezione vengono riportate le tecnologie richieste per il completamento del lavoro stagistico.
\begin{itemize}

	\item \textbf{Arduino IDE:} \\
	\begin{figure}[htbp]
	\centering
	\includegraphics[scale=.4]{./capitoli/capitolo2/img/ide}
	\caption{Esempio di IDE Arduino}
	\end{figure}
	\textit{Arduino IDE} è un semplice \textit{framework} di programmazione multi-piattaforma che permette di caricare nella scheda Arduino il codice sorgente che si vuole implementare. E' il \textit{software }ufficiale fornito da Arduino e permette un utilizzo semplice della piattaforma. L'interfaccia dell'IDE permette di lavorare su di uno \textit{sketch} nel quale si deve caricare tutto il codice che si intende implementare nella \textit{board}. Sono obbligatorie le dichiarazioni di due metodi fondamentali:
	\begin{itemize}
		\item \textbf{void setup()} : metodo che viene evocato ogni volta che la scheda viene riavviata, all'interno del quale devono essere dichiarate le modalità di utilizzo dei \textit{pin} connessi alla scheda;
		\item \textbf{void loop()} : metodo che viene ripetuto continuamente dalla scheda in un \textit{loop} infinito, finché viene mantenuta energia o non viene effettuato il \textit{reset}. 	
	
	\end{itemize}
	La sua semplicità rende questo strumento anche altrettanto limitato nella flessibilità di utilizzo. Per organizzare il codice in maniera strutturata per classi è necessario l'utilizzo di uno strumento secondario. L'IDE Arduino infatti permette la compilazione e l'\textit{upload} del codice presente solo nello \textit{sketch}. Senza l'utilizzo di un \textit{framework} esterno, bisognerebbe dichiarare tutte le classi all'interno di un unico \textit{file}.
	
	
	\item \textbf{Arduino UNO}:\\
	Come descritto in precedenza mi è stato richiesto l'utilizzo di una scheda Aduino UNO. Le specifiche tecniche di questa \textit{board} sono:
	\begin{itemize}
	\item \textit{Microcontroller}:	ATmega328;
	\item \textit{Operating Voltage}:	5V;
	\item \textit{Input Voltage (recommended)}: 7-12V;
	\item \textit{Input Voltage (limits)}:	6-20V;
	\item \textit{Digital I/O Pins	}: 14 (of which 6 provide PWM output);
	\item \textit{Analog Input Pins}: 6;
	\item \textit{DC Current per I/O Pin}:	40 mA;
	\item \textit{DC Current for 3.3V Pin}:	50 mA;
	\item \textit{SRAM}:	2 KB (ATmega328);
	\item \textit{EEPROM}:	512 Byte (ATmega328);
	\item \textit{Clock Speed}:	16 MHz;
	\item \textit{Length}:	68.6 mm;
	\item \textit{Width}:	53.4 mm;
	\item \textit{Weight}: 	25 g.
	\end{itemize}

	La scelta di questa scheda sono ricadute per i seguenti motivi:
	\begin{itemize}
	
	\item \textit{Affidabilità:} il prodotto 100\% italiano ha alle spalle una lunga storia di successi che hanno reso questo prodotto famoso per la sua affidabilità e robustezza;
	\item \textit{Feedback utenti:} poiché il prodotto è in commercio da qualche anno, i numerosi utenti che ne hanno fatto utilizzo hanno creato una solida rete di \textit{feedback} e risoluzione dei problemi;
	\item \textit{Riutilizzo software:} queste schede sono già state ampiamente usate per uno svariato numero di progetti con licenza gratuita, questo comporta la facile reperibilità \textit{online} di codice utile;
	\item  \textit{Componenti esterne:} molti produttori di componenti \textit{hardware} esterni, come sensori o schede di interfacce, forniscono librerie di interfacce perfettamente funzionanti e testati con l'ambiente Arduino;
	\item \textit{Prezzo:} 	
	il prezzo di una scheda Arduino si aggira sui 20 euro, mentre il microcontrollore ATmega328 dai 3 ai 5 euro. Se si utilizzassero solamente schede di questo tipo per il progetto, il risultato finale non risulterebbe economico. Durante la fase di codifica verranno usate due schede Arduino UNO per velocizzarne lo sviluppo, invece, il prodotto finale sarà composto da \textit{board} personalizzata che utilizzerà un microcontrollori con il codice già caricato, compatta ed economica.
	
	\end{itemize}


\item \textbf{Tiny-RTC (Real Time Clock Module)}\\
Componente \textit{hardware} per la misurazione del tempo. \textit{Tiny-RTC} non è altro che un modulo \textit{hardware} con la funzionalità di orologio configurabile tramite \textit{sketch} Arduino e provvisto di protocolli di I2C per l'interrogazione del modulo per il ritorno di data ed ora. 
\begin{figure}[htbp]
\centering
\includegraphics[scale=.3]{./capitoli/capitolo2/img/rtc}
\caption{Tiny RTC connessa a scheda Arduino UNO}

\end{figure}
 
Il modulo deve essere connesso alla \textit{board} Arduino tramite \textit{bus I2C} come mostrato nella figura 2.8. \\

Specifiche tecniche:
\begin{itemize}
    \item 3.0-5.5V input voltage
    \item Waterproof
    \item -55°C to+125°C temperature range
    \item ±0.5°C accuracy from -10°C to +85°C
    \item 1 Wire interface


\end{itemize}



Descrizione delle connessioni:

\begin{itemize}
\item \textbf{BAT}: 	Battery voltage 	
\item \textbf{GND}:	Ground 	Ground
\item \textbf{VCC}: Power the module and charge the battery
\item \textbf{SDA}: 	I2C data for the RTC
\item \textbf{SCL}:   I2C clock for the RTC
\item \textbf{DS}: 	Sensor output 	
\item \textbf{SQ}: 	Square wave output 
\end{itemize}
	
	
\item \textbf{Qt Creator}\\
\begin{figure}[htbp]
\centering
\includegraphics[scale=.8]{./capitoli/capitolo2/img/qt}
\caption{Logo Qt}
\end{figure}
Non mi è stato propriamente imposto l'utilizzo di \textit{Qt-Creator}, ma dal momento che mi era stato vivamente consigliato e durante il mio iter universitario ho avuto occasione di utilizzarlo frequentemente, è stato concordato sin dall'inizio il suo utilizzo per la codifica del codice.
\end{itemize}

\subsection{Vincoli Metodologici}

Come metodologie di lavoro mi sono state lasciate molte libertà, fatta eccezione per:

\begin{itemize}
	\item \textbf{Sistema di controllo di versionamento}\\
		\begin{figure}[htbp]
\centering
\begin{minipage}[c]{.45\textwidth}
	\centering\setlength{\captionmargin}{0pt}
	\includegraphics[width=0.50\textwidth]{./capitoli/capitolo2/img/github}
	\caption{Logo GitHub}
\end{minipage}
\hspace{10mm}
\begin{minipage}[c]{.45\textwidth}
	\centering\setlength{\captionmargin}{0pt}%
	\includegraphics[width=0.50\textwidth]{./capitoli/capitolo2/img/bitbucket}
	\caption{Logo Bitbucket}
\end{minipage}%

\end{figure}  
 Il responsabile del dipartimento mi ha richiesto l'utilizzo di \textit{git hub} come \textit{repository} del mio codice per la parte riguardante ARPAV. Invece il responsabile di FabLab ha richiesto l'utilizzo di un \textit{repository} di \textit{bit-bucket} privato per il codice da utilizzare durante il periodo di \textit{testing} dei dispositivi assieme al prototipo pluviometro.

\item \textbf{Rilascio continuo di prototipi}\\
	FabLab di Verona ha contemporaneamente molti progetti in corso d'opera, di cui buona parte assegnati sempre dal dipartimento di reti ed informatica di Padova. Con l'inizio del mio lavoro, l'officina ha dovuto spostare maggiormente il baricentro dell'attenzione sul progetto pluviometro. Per questo motivo mi è stato richiesto di organizzare il mio lavoro con modalità di progettazione e sviluppo attraverso \textit{prototipi}. 
	
\item \textbf{Relazioni avanzamento lavoro}\\
	Alla fine di ogni settimana ero tenuto ad aggiornare FabLab dei progressi ottenuti tramite una breve relazione da inviare via \textit{mail} direttamente al responsabile dell'officina.
\end{itemize}


\subsection{Vincoli Temporali}

L'Università di Padova rende obbligatorio lo svolgimento del lavoro di \textit{stage} nell'arco di 300 - 320 ore. Durante i colloqui conoscitivi è stato redatto un documento  di pianificazione del lavoro dove si sancivano le ore di lavoro effettive e come queste saranno suddivise. Il lavoro di \textit{stage} è stato dunque stabilito doversi svolgere in 320 nell'arco di 9 settimane. Un vincolo esterno importante da considerare è stata la necessità di avere sufficiente codice da avere un prototipo funzionante prima delle vacanze di Natale, da consegnare a \textit{FabLab} di Verona per poter iniziare i primi \textit{test} sul prototipo di pluviometro. 




\chapter{L'Attività di Stage}
\thispagestyle{fancy} 

[:] Descrizione introduttiva del capitolo e delle sue sezioni

\section{Pianificazione del Lavoro}

[:] Sezione in cui descrivo la pianificazione a monte di tutta l'attività di stage in relazione con i vincoli precedentemente assunti assieme all'azienda.



\subsection{Preparazione al Lavoro di Stage}

[:] Sezione in cui descrivo come mi sono preparato per affrontare il lavoro dello stage il meno impreparato possibile, così da poter attutire possibili ritardi.

\subsubsection{Studio del Dominio}

[:] Sottosezione in cui spiego come mi sono affacciato al dominio richiesto per lo stage, assumendone tutti i pro, i contro, possibilità e criticità che ho riscontrato all'inizio dell'attività di stage.

\subsubsection{Studio delle Tecnologie}

[:] Sottosezione in cui descrivo, uno per uno, le tecnologie e gli strumenti necessarie per intraprendere lo stage e i supporti utilizzati per acquisire tali conoscenze.


\subsubsection{Criticità riscontrate}

[:] Descrizione delle possibili criticità riscontrate a fronte degli obbiettivi richiesti in relazione con il dominio e le mie conoscenze di base.



\section{Analisi dei Requisiti}
[:] Introduzione alla sezione e breve descrizione delle sottosezioni


\subsection{Classificazione dei Requisiti}

[:] Descrizione delle suddivisioni fatte per classificare i requisiti in modo tale da risultare chiari e comprensibili.


\subsection{Individuazione dei Requisiti}

[:] Descrizione dei metodi utilizzati per individuare i requisiti e visualizzazione degli stessi tramite tabella

\subsubsection{Strumenti Utilizzati}

[:] Elenco e descrizione degli strumenti utilizzati per il tracciamento dei requisiti

\subsection{Casi d'Uso}

[:] Descrizione del motivo per cui i casi d'uso non sono stati usati come mi ero prefissato ad inizio stage.

\subsection{Dettagli Degni di Nota}

[:] Sezione in cui prendo in considerazione alcuni requisiti particolari analizzandone gli aspetti di difficoltà, problematiche e metodi di risoluzione del problema.

\section{Progettazione e Codifica}

[:] Descrizione dell'attività di Progettazione e delle scelte fatte durante lo stage di incrementare la progettazione di volta in volta.

\subsection{Introduzione Architettura Hardware}

[:] Sezione in cui descrivo la composizione delle componenti hardware durante l'attività di progettazione e codifica e come poi queste si siano unificate in un unico componente successivamente durante i test.

\subsection{Introduzione Architettura Software}

[:] Descrivo brevemente l'architettura software nel suo complesso delle due componenti hardware.

\subsection{Architettura Libreria Scheda Master}

[:] Sezione in cui descrivo l'architettura della parte riguardante la libreria Master che permette di interfacciarsi con la Scheda Slave facilmente. 


\subsection{Architettura Scheda Slave}

[:] Sezione in cui descrivo l'architettura della Scheda Slave, che si interfaccia direttamente con il Pluviometro (questa parte è più ampia della precedente).

\subsection{Design Pattern Utilizzati}

[:] Sezione in cui elenco, motivo e descrivo i Design Pattern utilizzati nell'architettura del progetto.

\subsection{Dettagli Degni di Nota}

[:] Sezione in cui espongo alcuni dettagli rilevanti dell'architettura e soprattutto le difficoltà avute a causa delle limitate risorse hardware delle schede in mio possesso.

\subsection{Progettazione di Dettaglio e Codifica}

[:] Introduzione della sezione in cui spiego come mi sono adattato alle continue richieste di prototipi da parte dell'azienda e come ho organizzato la progettazione in modo da rendere il codice facilmente modificabile e riutilizzabile per una codifica incrementale.

\subsubsection{Dettagli Degni di Nota}

[:] Sezione in cui descrivo alcuni metodi della libreria Master che permettono un facile interfacciamento con la scheda Slave. In aggiunta come sia possibile aggiungere facilmente nuove funzionalità al sistema senza alcuna iterazione, ma incrementando il codice.

\subsection{L'Utilizzo dei Prototipi}

[:] Breve sezione in cui spiego come, a causa delle necessità aziendali, mi sia trovato costretto ad un continuo ciclo di "Progettazione, Codifica, Test, Progettazione Incrementale, Codifica, Test..", approccio diverso da quello affrontato durante il mio percorso accademico.

\section{Verifica e Validazione}

[:] Descrizione dell'attività di verifica e validazione e breve descrizione delle sottosezioni successive

\subsection{Analisi Statica}

[:] Elenco e descrizione degli strumenti utilizzati per l'analisi statica e alcuni valori che meritano d'esser presi in considerazione.

\subsection{Test sul Sistema Slave}

[:] Sezione in cui descrivo in dettagli i test effettuati sulla scheda Slave, in progressione con gli incrementi effettuati.

\subsection{Test sul Sistema Master Slave}

[:] Descrizione dei test effettuati sul sistema Slave integrato con le interazioni con la scheda master in progressione con gli incrementi effettuati.

\subsection{Test di Sistema}

[:] Descrizione dei test effettuati sull'intero sistema.

\subsection{Dettagli Degni di Nota}

[:] Descrizione del problema degli impulsi "falsi positivi" inviati dal pluviometro, riscontrati durante i test dell'intero sistema e soluzione.

\subsection{Consuntivo Orario Finale}

[:] Descrizione oggettiva e motivazione fra le discrepanze di orario preventivato ed effettivo in relazione agli obbiettivi raggiunti


\chapter{Valutazioni Finali}
\thispagestyle{fancy} 

\section{Raggiungimento degli Obbiettivi}

[:] Descrizione della sezione e introduzione delle sue sottosezioni

\subsection{Rendiconto degli Obbiettivi}

[:] Descrizione oggettiva fra obbiettivi soddisfati e non.


\section{Difficoltà Riscontrate}

[:] Elenco e descrizione delle difficoltà avute durante il lavoro di Stage con relativa soluzione attuata.

\section{Bilancio Formativo}

[:] Descrizione della sezione e introduzione delle sue sottosezioni.

\subsection{Preparazione Iniziale}

[:] Descrizione delle mie capacità iniziali e come queste mi siano state utili per il periodo di stage; corsi utili per la mia attività specifica e cosa invece mi è mancato.

\subsection{Esperienze Acquisite}

[:] Elenco e descrizione esperienze acquisite durante il periodo di stage suddivise fra competenze e conoscenze.

\subsection{Considerazioni Finali}

[:] Sezione in cui riassumo l'esperienza di stage facendone un'analisi obbiettiva. Considerando le mie capacità iniziali e relazionandole con l'offerta proposta dall'azienda, le difficoltà riscontrate e la mia preparazione e capacità di analizzare e affrontare i problemi riscontrati. Come mi siano stati più utili alcuni corsi universitari e come invece altri, se fatti in modo diverso, avrebbero potuto conferirmi maggior competenze per la riuscita dello stage. Ed infine cosa mi ha trasmesso lo stage in termini di competenze per l'inserimento nel mondo del lavoro.


\newpage

\chapter{Glossario}
\label{Glossario}

Il glossario segue le seguenti regole topografiche:
\begin{enumerate}


	\item \textbf{Acronimi:} \textbf{{\color{OliveGreen}ASE }(Acronimo Solo d'Esempio):}\\ descrizione dell'acronimo;
	\item \textbf{Termini:} \textbf{{\color{Plum}Termine}:} \\
	descrizione del termine.

\end{enumerate}

\section*{A}

\begin{itemize}
	\item \textbf{{\color{OliveGreen}ARPAV} (Agenzia Regionale per la Prevenzione e Protezione Ambientale nel Veneto) :}\\
	 agenzia regionale le cui attività competenti riguardano la tutela, il controllo, il recupero dell'ambiente e per la prevenzione e promozione della salute collettiva al fine di conseguire la massima efficacia nell'individuazione e nella rimozione dei fattori di rischio per l'uomo e per l'ambiente;
	\item \textbf{{\color{Plum} Applicativi Web-Based}:} \\
	 programmi che funzionano su piattaforma Internet/Intranet e vengono visualizzati ed eseguiti da un browser qualsiasi. La sola installazione che richiedono è quella che viene eseguita su un unico computer dedicato chiamato Web-server
\end{itemize}

\section*{C}
\begin{itemize}

	

	\item \textbf{{\color{OliveGreen}CIVEN} (Coordinamento Interuniversitario Veneto per le Nanotecnologie) :}\\
	 associazione senza fini di lucro con lo scopo di progettare e realizzare iniziative di formazione, di ricerca, di sperimentazione industriale e di trasferimento tecnologico al mondo imprenditoriale nell'ambito del settore delle nanotecnologie;

	\item \textbf{{\color{Plum}Comitato provinciale di coordinamento }:} \\
	comitato il cui fine è di garantire il coordinamento delle attività del dipartimento provinciale di ARPAV con le attività delle competenti strutture della provincia e dei comuni. 
	Il comitato ha compiti di consulenza e di proposta; in particolare:
	\begin{itemize}
		\item formula al direttore generale dell’ARPAV proposte per la definizione dei programmi annuali di attività;
		\item verifica l'andamento e i risultati delle attività programmate, esprimendo al direttore generale dell'ARPAV valutazioni e proposte. 
	\end{itemize}
	

\end{itemize}


\section*{G}
\begin{itemize}

	\item \textbf{{\color{OliveGreen}GPP} (Green Public Procurement) :} \\
	modalità di acquisto, rivolta e adottata dalle pubbliche amministrazioni locali o nazionali, basata su criteri ambientali oltre che sulla qualità e prezzo di prodotti e servizi. 
	GPP significa:
	\begin{itemize}
		\item acquistare solo ciò che è indispensabile;
	 	\item considerare l’impatto ambientale del prodotto lungo tutto il suo ciclo di vita e non solo al momento dell’utilizzo;
    		\item stimolare in senso ambientalmente sostenibile l’innovazione di prodotti e servizi;
    		\item adottare comportamenti d’acquisto responsabili e dare il buon esempio nei confronti dei cittadini;
    		\item sviluppare la comunicazione e lo scambio di informazioni tra gli enti locali, le imprese e i consumatori.
	\end{itemize}

\end{itemize}

\section*{I}
\begin{itemize}

	\item \textbf{{\color{OliveGreen}IPPC} ( Integrated Pollution Prevention and Control) :} \\
	 direttiva che prevede un nuovo approccio per la riduzione degli impatti ambientali con la graduale applicazione di un insieme di soluzioni tecniche (impiantistiche, gestionali e di controllo) per evitare, o, qualora non sia possibile, ridurre, le emissioni di inquinanti nell'aria, nell'acqua e nel suolo, comprese misure relative ai rifiuti.

\end{itemize}

\section*{M}
\begin{itemize}
	\item \textbf{{\color{Plum} Map-Server}:} \\
	ambiente multipiattaforma di sviluppo e fruizione \textit{Open Source} finalizzato alla rappresentazione di dati geospaziali;
\end{itemize}

\section*{O}

\begin{itemize}

	\item \textbf{{\color{OliveGreen}OIV} (Organismo Indipendente di Valutazione):} \\	organismo, ai sensi dell'articolo 14 del decreto legislativo n. 150/2009, che svolge, all'interno di ciascuna amministrazione, un ruolo fondamentale nel processo di misurazione e valutazione delle strutture e dei dirigenti e nell'adempimento degli obblighi di integrità e trasparenza posti alle amministrazioni.
	
\end{itemize}

\section*{P}

\begin{itemize}
\item \textbf{{\color{OliveGreen}PDCA} (\textit{Plan Do Check Act)}:} \\
modello studiato per il miglioramento continuo della qualità in un'ottica a lungo raggio. Serve per promuovere una cultura della qualità che è tesa al miglioramento continuo dei processi e all'utilizzo ottimale delle risorse. Questo strumento parte dall'assunto che per il raggiungimento del massimo della qualità sia necessaria la costante interazione tra ricerca, progettazione, test, produzione e vendita. Per migliorare la qualità e soddisfare il cliente, le quattro fasi devono ruotare costantemente, tenendo come criterio principale la qualità.

\end{itemize}

\section*{R}
\begin{itemize}

\item \textbf{{\color{OliveGreen}RE.S.M.I.A.} (Reti e Stazioni di Monitoraggio Innovative per l'Ambiente) :}\\
progetto di ricerca industriale con l'obiettivo di potenziare ed integrare la rete di monitoraggio ambientale a disposizione di ARPAV;

\end{itemize}

\section*{S}

\begin{itemize}

	\item \textbf{{\color{Plum} Subcontractor}:}\\
	situazione che si verifica quando ad un'azienda o un ente viene delegato parte di un lavoro già sotto contratto;

\end{itemize}

\section*{W}
\begin{itemize}
	\item \textbf{{\color{Plum}Web-Based}:} \\
	in informatica si intende per Web-Based qualsiasi cosa, applicazione, sistema informatico, \textit{hardware}, accessibile/fruibile via \textit{web} per mezzo di un \textit{network}.
	
	\item \textbf{{\color{OliveGreen} WSN} (Wireless Sensor Network):} \\
	 indica una determinata tipologia di rete che, caratterizzata da una architettura distribuita, è realizzata da un insieme di dispositivi elettronici autonomi in grado di prelevare dati dall'ambiente circostante e di comunicare tra loro;


\end{itemize}




\end{document}