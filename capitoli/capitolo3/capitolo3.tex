\chapter{L'Attività di Stage}
\thispagestyle{fancy} 

[:] Descrizione introduttiva del capitolo e delle sue sezioni

\section{Pianificazione del Lavoro}

[:] Sezione in cui descrivo la pianificazione a monte di tutta l'attività di stage in relazione con i vincoli precedentemente assunti assieme all'azienda.



\subsection{Preparazione al Lavoro di Stage}

[:] Sezione in cui descrivo come mi sono preparato per affrontare il lavoro dello stage il meno impreparato possibile, così da poter attutire possibili ritardi.

\subsubsection{Studio del Dominio}

[:] Sottosezione in cui spiego come mi sono affacciato al dominio richiesto per lo stage, assumendone tutti i pro, i contro, possibilità e criticità che ho riscontrato all'inizio dell'attività di stage.

\subsubsection{Studio delle Tecnologie}

[:] Sottosezione in cui descrivo, uno per uno, le tecnologie e gli strumenti necessarie per intraprendere lo stage e i supporti utilizzati per acquisire tali conoscenze.


\subsubsection{Criticità riscontrate}

[:] Descrizione delle possibili criticità riscontrate a fronte degli obbiettivi richiesti in relazione con il dominio e le mie conoscenze di base.



\section{Analisi dei Requisiti}
[:] Introduzione alla sezione e breve descrizione delle sottosezioni


\subsection{Classificazione dei Requisiti}

[:] Descrizione delle suddivisioni fatte per classificare i requisiti in modo tale da risultare chiari e comprensibili.


\subsection{Individuazione dei Requisiti}

[:] Descrizione dei metodi utilizzati per individuare i requisiti e visualizzazione degli stessi tramite tabella

\subsubsection{Strumenti Utilizzati}

[:] Elenco e descrizione degli strumenti utilizzati per il tracciamento dei requisiti

\subsection{Casi d'Uso}

[:] Descrizione del motivo per cui i casi d'uso non sono stati usati come mi ero prefissato ad inizio stage.

\subsection{Dettagli Degni di Nota}

[:] Sezione in cui prendo in considerazione alcuni requisiti particolari analizzandone gli aspetti di difficoltà, problematiche e metodi di risoluzione del problema.

\section{Progettazione e Codifica}

[:] Descrizione dell'attività di Progettazione e delle scelte fatte durante lo stage di incrementare la progettazione di volta in volta.

\subsection{Introduzione Architettura Hardware}

[:] Sezione in cui descrivo la composizione delle componenti hardware durante l'attività di progettazione e codifica e come poi queste si siano unificate in un unico componente successivamente durante i test.

\subsection{Introduzione Architettura Software}

[:] Descrivo brevemente l'architettura software nel suo complesso delle due componenti hardware.

\subsection{Architettura Libreria Scheda Master}

[:] Sezione in cui descrivo l'architettura della parte riguardante la libreria Master che permette di interfacciarsi con la Scheda Slave facilmente. 


\subsection{Architettura Scheda Slave}

[:] Sezione in cui descrivo l'architettura della Scheda Slave, che si interfaccia direttamente con il Pluviometro (questa parte è più ampia della precedente).

\subsection{Design Pattern Utilizzati}

[:] Sezione in cui elenco, motivo e descrivo i Design Pattern utilizzati nell'architettura del progetto.

\subsection{Dettagli Degni di Nota}

[:] Sezione in cui espongo alcuni dettagli rilevanti dell'architettura e soprattutto le difficoltà avute a causa delle limitate risorse hardware delle schede in mio possesso.

\subsection{Progettazione di Dettaglio e Codifica}

[:] Introduzione della sezione in cui spiego come mi sono adattato alle continue richieste di prototipi da parte dell'azienda e come ho organizzato la progettazione in modo da rendere il codice facilmente modificabile e riutilizzabile per una codifica incrementale.

\subsubsection{Dettagli Degni di Nota}

[:] Sezione in cui descrivo alcuni metodi della libreria Master che permettono un facile interfacciamento con la scheda Slave. In aggiunta come sia possibile aggiungere facilmente nuove funzionalità al sistema senza alcuna iterazione, ma incrementando il codice.

\subsection{L'Utilizzo dei Prototipi}

[:] Breve sezione in cui spiego come, a causa delle necessità aziendali, mi sia trovato costretto ad un continuo ciclo di "Progettazione, Codifica, Test, Progettazione Incrementale, Codifica, Test..", approccio diverso da quello affrontato durante il mio percorso accademico.

\section{Verifica e Validazione}

[:] Descrizione dell'attività di verifica e validazione e breve descrizione delle sottosezioni successive

\subsection{Analisi Statica}

[:] Elenco e descrizione degli strumenti utilizzati per l'analisi statica e alcuni valori che meritano d'esser presi in considerazione.

\subsection{Test sul Sistema Slave}

[:] Sezione in cui descrivo in dettagli i test effettuati sulla scheda Slave, in progressione con gli incrementi effettuati.

\subsection{Test sul Sistema Master Slave}

[:] Descrizione dei test effettuati sul sistema Slave integrato con le interazioni con la scheda master in progressione con gli incrementi effettuati.

\subsection{Test di Sistema}

[:] Descrizione dei test effettuati sull'intero sistema.

\subsection{Dettagli Degni di Nota}

[:] Descrizione del problema degli impulsi "falsi positivi" inviati dal pluviometro, riscontrati durante i test dell'intero sistema e soluzione.

\subsection{Consuntivo Orario Finale}

[:] Descrizione oggettiva e motivazione fra le discrepanze di orario preventivato ed effettivo in relazione agli obbiettivi raggiunti