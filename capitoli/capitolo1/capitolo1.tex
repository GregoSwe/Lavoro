\chapter{Profilo dell'Agenzia}
\label{1.0}
\thispagestyle{fancy} 

ARPAV\ped{g} è l'agenzia regionale per la prevenzione e protezione ambientale del Veneto, operativa dal 3 Ottobre 1997 in seguito alla Legge Regionale n32° del 18 Ottobre 1996.

\begin{figure}[htbp]
	\centering
	\includegraphics[scale=0.7]{./capitoli/capitolo1/img/logoARPAV.jpg}
	\caption{logo dell'agenzia}
\end{figure}

Le attività competenti riguardano la tutela, il controllo, il recupero dell'ambiente e per la prevenzione e promozione della salute collettiva al fine di conseguire la massima efficacia nell'individuazione e nella rimozione dei fattori di rischio per l'uomo e per l'ambiente. Le funzioni principale dell'agenzia riguardano attività tecnico-scientifiche per il monitoraggio, tutela e prevenzione di acqua, aria (inquinamento acustico ed elettromagnetico negli ambienti di vita), suolo, rifiuti solidi e liquidi, radioattività ambientale ed infine ai rischi di incidenti rilevanti attività industriali. L'esercizio delle attività di monitoraggio e prevenzione vengono effettuate in coordinazione con le unità locali socio sanitarie.\\
L'agenzia è suddivisa in vari organi operativi, i quali hanno funzionalità specifiche a seconda del ruolo che ricoprono. La suddivisione degli incarichi e delle competenze e la corretta comunicazione fra i vari dipartimenti, permette di gestire questo vasto ente in modo efficiente e sistematico.\\
Lo \textit{stage} in oggetto è stato svolto presso il servizio informatico e di reti in via Cairoli, 4/d. In questa sede vengono svolte le mansioni per la gestione dell'infrastruttura informatica e delle risorse strumentali hardware di tutta l'agenzia, fra cui:
\begin{itemize}
	\item gestione e coordinamento delle banche dati dell'agenzia;
	\item assistenza sulle applicazioni informatiche dell'agenzia;
	\item definizione degli indicatori ambientali e dei rapporti;
	\item fornitura degli standard operativi, architetture delle realizzazioni, attivazione e gestione tecnica dei portali internet/intranet;
	\item gestione connettività aziendale, voce dati;
	\item gestione tecnico operativa per il funzionamento, la manutenzione e la connettività delle reti di monitoraggio dell'azienda.
\end{itemize} 


\section{Cosa offre: Prodotti e Servizi}

[:] descrizione della sezione.

\subsection{I Prodotti di ARPAV}

[:] lista generale dei prodotti generati da ARPAV con breve descrizione.

\subsection{I Servizi di ARPAV}

[:] lista dei servizi offerti da ARPAV con breve descrizione.

\section{Organizzazione Interna}

[:] Descrizione della sezione con breve introduzione alle sottosezioni.

\subsection{Processi di Sviluppo}

[:] Descrizione dei processi attuati all'interno dell'agenzia per lo sviluppo dei prodotti e la fruizione dei servizi e i metodi utilizzati per i processi di sviluppo.

\subsection{Metodologie di Supporto ai Processi}

[:] Elenco e descrizione delle metodologie utilizzate a supporto dei processi dall'agenzia.

\subsection{Strumenti di Supporto ai Processi}

[:] Elenco e descrizione degli strumenti utilizzati dall'agenzia a supporto dei processi.

\subsection{Tecnologie Utilizzate}

[:] Elenco e descrizione delle tecnologie utilizzate dall'agenzia in relazione all'ambito in cui vengono applicate.

\section{Relazioni Esterne}

[:] Descrizione dei clienti diretti dell'agenzia e la metodologia di interazione; breve descrizione dei clienti secondari che usufruiscono dei prodotti erogati dall'agenzia.

\subsection{Clienti Diretti}

[:]  descrizione particolare dei clienti diretti dei servizi e prodotti dell'agenzia e le metodologie di relazione con essi

\subsection{Clienti Secondari}

[:] descrizione dei clienti secondari, ma spesso finali dei prodotti e servizi forniti dall'agenzia.

\subsection{Orientamento all'Innovazione}

[:] Descrizione della propensione dell'ente regionale all'innovazione in relazione con i suoi prodotti e servizi e progetti futuri.

