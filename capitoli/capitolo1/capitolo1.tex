\chapter{Profilo dell'Agenzia}
\label{1.0}
\thispagestyle{fancy} 

ARPAV\ped{g} è l'agenzia regionale per la prevenzione e protezione ambientale del Veneto, operativa dal 3 Ottobre 1997 in seguito alla Legge Regionale n32° del 18 Ottobre 1996.

\begin{figure}[htbp]
	\centering
	\includegraphics[scale=0.7]{./capitoli/capitolo1/img/logoARPAV.jpg}
	\caption{logo dell'agenzia}
\end{figure}

Le attività competenti riguardano la tutela, il controllo, il recupero dell'ambiente e per la prevenzione e promozione della salute collettiva al fine di conseguire la massima efficacia nell'individuazione e nella rimozione dei fattori di rischio per l'uomo e per l'ambiente. Le funzioni principale dell'agenzia riguardano attività tecnico-scientifiche per il monitoraggio, tutela e prevenzione di acqua, aria (inquinamento acustico ed elettromagnetico negli ambienti di vita), suolo, rifiuti solidi e liquidi, radioattività ambientale ed infine ai rischi di incidenti rilevanti attività industriali. L'esercizio delle attività di monitoraggio e prevenzione vengono effettuate in coordinazione con le unità locali socio sanitarie.\\
L'agenzia è suddivisa in vari organi operativi, i quali hanno funzionalità specifiche a seconda del ruolo che ricoprono. La suddivisione degli incarichi e delle competenze e la corretta comunicazione fra i vari dipartimenti, permette di gestire questo vasto ente in modo efficiente e sistematico.\\
Lo \textit{stage} in oggetto è stato svolto presso il servizio informatico e di reti in via Cairoli, 4/d. In questa sede vengono svolte le mansioni per la gestione dell'infrastruttura informatica e delle risorse strumentali hardware di tutta l'agenzia.

\begin{table}[htbp]
\centering
\begin{tabular}{|p{0.95\textwidth}|}
\hline

\begin{itemize}

    \item gestione e coordinamento delle banche dati dell'agenzia;
	
	\item assistenza sulle applicazioni informatiche dell'agenzia;
	
	\item definizione degli indicatori ambientali e dei rapporti;
	
	\item fornitura degli standard operativi, architetture delle realizzazioni, attivazione e gestione tecnica dei portali internet/intranet;
	
	\item gestione connettività aziendale, voce dati;
	
	\item gestione tecnico operativa per il funzionamento, la manutenzione e la connettività delle reti di monitoraggio dell'azienda.
\end{itemize}
	\\
	
\hline
\end{tabular}
\caption{Mansioni dipartimento servizio informatico e reti}
\end{table}

\section{Cosa offre: Prodotti e Servizi}

In questa sezione vengono descritti i prodotti e i servizi che ARPAV mette a disposizione, ponendo particolare attenzione alla sede dello stage.

\subsection{I Prodotti di ARPAV}

Non essendo un'azienda a scopo di lucro, ma un'agenzia regionale tenuta alla parità di bilancio, l'orientamento generale non è propenso alla distribuzione e vendita di prodotti. ARPAV, per lo più, collabora con altre aziende o enti per realizzazioni di progetti in ambito ambientale,  come \textit{patner, subcontractor\ped{g}} o \textit{leader}.



\begin{longtable}{ p{0.2\textwidth} | p{0.3\textwidth} | p{0.5\textwidth}}
\toprule
 Logo &  Nome &  Descrizione \tabularnewline
\midrule
\vfill \includegraphics[scale=0.8]{./capitoli/capitolo1/img/alpini} & \vfill \textbf{{\color{ForestGreen}SedAlp}: Programma Spazio Alpino} &
 Sviluppo e \textit{testing} di politiche e strumenti utili alla gestione integrata del trasporto di sedimenti nei bacini alpini al fine di ridurre il rischio legato al trasporto solido e allo stesso tempo di migliorare la condizione ecologica degli ambienti acquatici e ripararli e ridurre l'impatto ambientale creato dalle centrali idroelettriche. \\
\midrule
\vfill \includegraphics[scale=0.7]{./capitoli/capitolo1/img/med} & \vfill \textbf{{\color{ForestGreen}CAIMANs}: Programma Med}  & Valutando l'impatto sulla qualità dell'aria da parte delle navi crociera e in generale delle navi passeggeri, il progetto mira a porre le basi per l'identificazione dei punti critici e per proporre orientamenti per futuri progetti e politiche trasnazionali che affrontino la mitigazione dell'inquinamento atmosferico dovuto al traffico navale passeggeri. \\
\midrule
\vfill \includegraphics[scale=0.7]{./capitoli/capitolo1/img/park} & \vfill \textbf{ {\color{ForestGreen}GuardEn}: Programma South East Europe } & sviluppo e \textit{testing} di un possibile quadro di riferimento finalizzato al supporto di un programma di implementazione di locali strategie per la gestione e prevenzione del rischio ambientale legato all'attività agricola e agroalimentare. In particolare per i territori interessati dall'inquinamento del suolo e dell'acqua, da proporre per l'applicazione alle aziende del settore. \\
\midrule
\vfill \includegraphics[scale=0.7]{./capitoli/capitolo1/img/resmia} & \vfill \textbf{{\color{ForestGreen}3PClim}: Programma Interreg IV Italia – Austria} &   aggiornamento della climatologia delle Alpi orientali, con la produzioni di cartografie tematiche, elaborazioni e proiezioni climatiche. \\
\bottomrule

\caption{Progetti ARPAV}
\end{longtable}

[:] lista generale dei prodotti generati da ARPAV con breve descrizione.

\subsection{I Servizi di ARPAV}

[:] lista dei servizi offerti da ARPAV con breve descrizione.

\section{Organizzazione Interna}

[:] Descrizione della sezione con breve introduzione alle sottosezioni.

\subsection{Processi di Sviluppo}

[:] Descrizione dei processi attuati all'interno dell'agenzia per lo sviluppo dei prodotti e la fruizione dei servizi e i metodi utilizzati per i processi di sviluppo.

\subsection{Metodologie di Supporto ai Processi}

[:] Elenco e descrizione delle metodologie utilizzate a supporto dei processi dall'agenzia.

\subsection{Strumenti di Supporto ai Processi}

[:] Elenco e descrizione degli strumenti utilizzati dall'agenzia a supporto dei processi.

\subsection{Tecnologie Utilizzate}

[:] Elenco e descrizione delle tecnologie utilizzate dall'agenzia in relazione all'ambito in cui vengono applicate.

\section{Relazioni Esterne}

[:] Descrizione dei clienti diretti dell'agenzia e la metodologia di interazione; breve descrizione dei clienti secondari che usufruiscono dei prodotti erogati dall'agenzia.

\subsection{Clienti Diretti}

[:]  descrizione particolare dei clienti diretti dei servizi e prodotti dell'agenzia e le metodologie di relazione con essi

\subsection{Clienti Secondari}

[:] descrizione dei clienti secondari, ma spesso finali dei prodotti e servizi forniti dall'agenzia.

\subsection{Orientamento all'Innovazione}

[:] Descrizione della propensione dell'ente regionale all'innovazione in relazione con i suoi prodotti e servizi e progetti futuri.

