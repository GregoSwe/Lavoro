\chapter{Profilo dell'Agenzia}
\label{1.0}
\thispagestyle{fancy} 

ARPAV\ped{g} è l'agenzia regionale per la prevenzione e protezione ambientale del Veneto, operativa dal 3 Ottobre 1997 in seguito alla Legge Regionale n32° del 18 Ottobre 1996.

\begin{figure}[htbp]
	\centering
	\includegraphics[scale=0.7]{./capitoli/capitolo1/img/logoARPAV.jpg}
	\caption{logo dell'agenzia}
\end{figure}

Le attività competenti riguardano la tutela, il controllo, il recupero dell'ambiente e per la prevenzione e promozione della salute collettiva al fine di conseguire la massima efficacia nell'individuazione e nella rimozione dei fattori di rischio per l'uomo e per l'ambiente. Le funzioni principale dell'agenzia riguardano attività tecnico-scientifiche per il monitoraggio, tutela e prevenzione di acqua, aria (inquinamento acustico ed elettromagnetico negli ambienti di vita), suolo, rifiuti solidi e liquidi, radioattività ambientale ed infine ai rischi di incidenti rilevanti attività industriali. L'esercizio delle attività di monitoraggio e prevenzione vengono effettuate in coordinazione con le unità locali socio sanitarie.\\
L'agenzia è suddivisa in vari organi operativi, i quali hanno funzionalità specifiche a seconda del ruolo che ricoprono. La suddivisione degli incarichi e delle competenze e la corretta comunicazione fra i vari dipartimenti, permette di gestire questo vasto ente in modo efficiente e sistematico.\\
Lo \textit{stage} in oggetto è stato svolto presso il servizio informatico e di reti in via Cairoli, 4/d. In questa sede vengono svolte le mansioni per la gestione dell'infrastruttura informatica e delle risorse strumentali hardware di tutta l'agenzia.

\begin{table}[htbp]
\centering
\begin{tabular}{|p{0.95\textwidth}|}
\hline

\begin{itemize}

    \item gestione e coordinamento delle banche dati dell'agenzia;
	
	\item assistenza sulle applicazioni informatiche dell'agenzia;
	
	\item definizione degli indicatori ambientali e dei rapporti;
	
	\item fornitura degli standard operativi, architetture delle realizzazioni, attivazione e gestione tecnica dei portali internet/intranet;
	
	\item gestione connettività aziendale, voce dati;
	
	\item gestione tecnico operativa per il funzionamento, la manutenzione e la connettività delle reti di monitoraggio dell'azienda.
\end{itemize}
	\\
	
\hline
\end{tabular}
\caption{Mansioni dipartimento servizio informatico e reti}
\end{table}

\section{Cosa offre: Prodotti e Servizi}

In questa sezione vengono elencati e descritti le tipologie di prodotti che ARPAV produce e i tipi diversi di servizi che offre, ponendo particolare attenzione a ciò che viene erogato dalla sede dello \textit{stage}.

\subsection{I Prodotti di ARPAV}

Non essendo un'azienda a scopo di lucro, ma un'agenzia regionale, ARPAV è tenuta alla parità di bilancio. L'orientamento generale non è propenso alla distribuzione e vendita di prodotti quindi, ARPAV, per lo più, collabora con altre aziende o enti per realizzare progetti in ambito ambientale come \textit{patner, subcontractor\ped{g}} o \textit{leader}.



\begin{longtable}{ p{0.2\textwidth} | p{0.3\textwidth} | p{0.5\textwidth}}

\textbf{Logo}& \textbf{Ruolo \& Nome}&  \textbf{Descrizione}\\

 \endhead
\midrule
\vfill \includegraphics[scale=0.8]{./capitoli/capitolo1/img/alpini} & \vfill \textbf{{\color{Plum}Ruolo}: \textit{Patner}} \newline \vfill \textbf{{\color{ForestGreen}SedAlp}: Programma Spazio Alpino}  &
 Sviluppo e \textit{testing} di politiche e strumenti utili alla gestione integrata del trasporto di sedimenti nei bacini alpini al fine di ridurre il rischio legato al trasporto solido e allo stesso tempo di migliorare la condizione ecologica degli ambienti acquatici e ripararli e ridurre l'impatto ambientale creato dalle centrali idroelettriche. \\
\midrule
\vfill \includegraphics[scale=0.7]{./capitoli/capitolo1/img/med} & \vfill \textbf{{\color{Plum}Ruolo}: \textit{Leader}} \newline \vfill \textbf{{\color{ForestGreen}CAIMANs}: Programma Med}  & Valutando l'impatto sulla qualità dell'aria da parte delle navi crociera e in generale delle navi passeggeri, il progetto mira a porre le basi per l'identificazione dei punti critici e per proporre orientamenti per futuri progetti e politiche trasnazionali che affrontino la mitigazione dell'inquinamento atmosferico dovuto al traffico navale passeggeri. \\
\midrule
\vfill \includegraphics[scale=0.7]{./capitoli/capitolo1/img/park} & \vfill \textbf{{\color{Plum}Ruolo}: \textit{Patner}} \newline \vfill \textbf{ {\color{ForestGreen}GuardEn}: Programma South East Europe } & Sviluppo e \textit{testing} di un possibile quadro di riferimento finalizzato al supporto di un programma di implementazione di locali strategie per la gestione e prevenzione del rischio ambientale legato all'attività agricola e agroalimentare. In particolare per i territori interessati dall'inquinamento del suolo e dell'acqua, da proporre per l'applicazione alle aziende del settore. \\
\midrule
\vfill \includegraphics[scale=0.7]{./capitoli/capitolo1/img/interr} & \vfill \textbf{{\color{Plum}Ruolo}: \textit{Patner}} \newline \vfill \textbf{{\color{ForestGreen}3PClim}: Programma Interreg IV Italia – Austria} &   Aggiornamento della climatologia delle Alpi orientali, con la produzioni di cartografie tematiche, elaborazioni e proiezioni climatiche. \\
\midrule
\vfill \includegraphics[scale=0.7]{./capitoli/capitolo1/img/resmia} & \vfill \textbf{{\color{Plum}Ruolo}: \textit{Patner}} \newline \vfill \textbf{{\color{ForestGreen}RE.S.M.I.A.}: Programma POR – FESR Veneto} & Progetto pilota di ricerca industriale con l’obiettivo di potenziare ed integrare la rete di monitoraggio ambientale a disposizione di ARPAV.\\
\bottomrule
\caption{Progetti ARPAV}
\end{longtable}

Il dipartimento di informatica e reti, in collaborazione con CIVEN\ped{g}, sta seguendo in particolare il progetto \textbf{RE.S.M.I.A.}\ped{g}. Il progetto consiste nel potenziamento dell'infrastruttura delle stazioni di monitoraggio ambientale attualmente a disposizione di ARPAV, con la progettazione ed installazione di sensori. Il nuovo concetto di stazione farà uso di tecnologia WSN\ped{g}, applicativi \textit{Web-Based}\ped{g} e sarà caratterizzato dall'ottimizzazione della efficienza energetica del sistema. \\


Gli obbiettivi principali del progetto sono:
\begin{itemize}

	\item \textbf{minimizzazione dell'impatto ambientale dall'installazione delle stazioni di monitoraggio:} utilizzo di tecnologie per l'auto-alimentazione dell'impianto e utilizzo di tecnologie \textit{wireless} per ridurre al minimo l'invasione ambientale e con sperimentazione di nuovi sensori elettrochimici nanostrutturati per il monitoraggio in sito di metalli pesanti;
	\item \textbf{riduzione dei costi di produzioni ed installazione:} utilizzo di nuovi \textit{hardware} a basso costo con ottime prestazioni, bassa necessità di manutenzione e semplice installazione, anche in presenza di condizioni morfologiche del territorio estremamente critiche. Realizzazione di prodotti in grado di essere installati da qualunque persona, in particolare volontari sprovvisti di preparazione;
	\item \textbf{tutela dell'ambiente:} monitoraggio delle matrici ambientali anticipando anche le disposizioni legislative e ponendo attenzione alle tematiche sentite dall'opinione pubblica in tema di salute, quale ad esempio il monitoraggio di nanoparticelle in aria;
	\item \textbf{prevenzione dei rischi:} determinazione delle soglie minime di allarme dei parametri ambientali reperibili in tempi brevi per l'intero territorio della Regione del Veneto;

\end{itemize}

\begin{figure}[htbp]
\centering
\includegraphics[scale=0.3]{./capitoli/capitolo1/img/retiresmia}
\caption{Schema Reti RESMIA}
\end{figure}

Il dipartimento di informatica e reti ha il compito di incrementare la struttura del sistema di salvataggio dei dati e di potenziare l'interfaccia \textit{web} che rappresenta graficamente su un'opportuna mappa la dislocazione dei sensori e ed effettuare opportune interrogazioni sui dati forniti tramite \textit{Map Server}\ped{g}.

\subsection{I Servizi di ARPAV}

L'agenzia regionale offre una vasta gamma di servizi, i quali possono essere suddivisi in due macro categorie: servizi ambientali e servizi online. I primi, offrono un servizio su richiesta o erogati da ARPAV, i secondi invece sono servizi passivi offerti dal portale \textit{internet}, i cui fruitori possono accedervi tramite \textit{network}.

\subsubsection{Servizi Ambientali}

\begin{longtable}{ p{0.3\textwidth} | p{0.3\textwidth} | p{0.4\textwidth}}
\textbf{Nome}& \textbf{Possibili Fruitori}&  \textbf{Descrizione} \\
\endhead

	\midrule
	\textbf{{\color{OliveGreen}Acquisti pubblici verdi-GPP\ped{g}}} & Pubbliche amministrazioni locali o nazionali  & Informazione delle pubbliche amministrazioni circa l'adozione di pratiche d'acquisto verdi che riducono l'uso di risorse naturali, la produzione di rifiuti, i rischi ambientali \\
	\midrule
	\textbf{{\color{OliveGreen}Certificazioni ambientali}} & Imprese soprattutto piccole e medie &  Diffusione all'interno del mondo produttivo di una nuova cultura di sistema per la gestione consapevole ed ecocompatibile dell'ambiente attraverso lo sviluppo di progetti, strumenti, protocolli ad hoc\\
	\midrule
	\textbf{{\color{OliveGreen}Comunicazione}} & Cittadini & Promozione delle attività di educazione ed informazione ambientale dei cittadini \\
	\midrule
	\textbf{{\color{OliveGreen}Progetti \& Cooperazione}} & Aziende, imprese, enti pubblici o regioni & Avvio e realizzazione di progetti, avvio di relazioni internazionali generalmente finanziati da fondi dell'Unione Europea \\
	\midrule
	\textbf{{\color{OliveGreen} Grandi opere}} & Aziende coinvolte in appalti pubblici in Veneto & attività di audit preventivo e di monitoraggio ambientale per garantire la compatibilità ambientale, il corretto inserimento dal punto di vista urbanistico, ambientale, trasportistico e sociale delle Grandi Opere \\
	\midrule
	\textbf{{\color{OliveGreen} Educazione per la sostenibilità}} & Chiunque &  Attività di educazione, informazione e comunicazione ambientale, protezione della natura al fine di promuovere e sviluppare comportamenti sostenibili \\
	\midrule
	\textbf{{\color{OliveGreen} IPPC\ped{g} e  Servizi alle aziende}} & Aziende ed Imprese & Consulenze sul piano di monitoraggio e controllo in fase istruttoria per il rilascio dell'autorizzazione integrata ambientale, ispezioni integrate ambientali nelle aziende IPPC del Veneto\\
	\midrule
	\textbf{{\color{OliveGreen} Pronta disponibilità}} &  Dipartimenti di Prevenzione delle ULSS regionali Organi di polizia giudiziaria &  Attività di analisi immediata di aria, acqua e suolo secondo le modalità previste\\
	\midrule
	\textbf{{\color{OliveGreen} Rischio industriale}} & Industrie & Individuazione, classificazione e probabilità dei pericoli provenienti dalle industrie che utilizzano o detengono sostanze chimiche per le loro attività \\
	\midrule
	\textbf{{\color{OliveGreen} Sicurezza impiantistica}} & Comuni, ASL, Prefettura, Procura & Verifica della corretta funzionalità di impianti e macchinari installati in ambienti di lavoro o di vita e soggetti a controlli periodici \\
	\bottomrule
	

\caption{Servizi Ambientali ARPAV}
\end{longtable}

\subsubsection{Servizi Online}

\begin{longtable}{p{0.3\textwidth}|p{0.7\textwidth}}
\textbf{Nome} & \textbf{Descrizione} \\
\endhead

\midrule
\textbf{{\color{Plum} Accesso informazioni ambientali}} & Accesso del pubblico all'informazione ambientale detenuta o prodotta da soggetti pubblici avviene anche mediante l'utilizzo delle tecnologie informatiche e dei mezzi di telecomunicazione \\
\midrule
\textbf{{\color{Plum} Glossari Ambientali}} & Strumento di informazione aggiornata ed esaustiva che ARPAV mette a disposizione dei cittadini per favorire la comprensione di termini 'ambientali' maggiormente utilizzati \\
\midrule
\textbf{{\color{Plum} Iscrizione bollettini}} & Iscrizione alla mailing list consente di ricevere i bollettini Meteo direttamente nella propria casella di posta elettronica. Meteo Veneto, Dolomiti Meteo, Meteo Spiagge e Meteo Garda i bollettini per cui è disponibile il servizio \\
\midrule
\textbf{{\color{Plum} Iscrizione bollettini via sms}} & Sottoscrivendo un abbonamento è possibile ricevere via sms i contenuti di alcuni bollettini prodotti per l'area delle Dolomiti. Dolomiti meteo e Dolomiti Neve e Valanghe i bollettini per i quali è disponibile il servizio \\
\midrule
\textbf{{\color{Plum} Iscrizione \textit{newsletter}}} &  E' possibile ricevere periodicamente nella propria casella di posta le \textit{newsletter} con informazioni su eventi, contenuti e attività \\
\midrule
\textbf{{\color{Plum} Iscrizione applicativo web ORSO }} &  Programma per il monitoraggio del flusso dei rifiuti attraverso le Regioni d'Italia, con standard di riferimento comuni che garantiscano rappresentatività delle informazioni raccolte, oltre ad agevolare lo scambio di informazioni finalizzato alla corretta gestione dei rifiuti per i Comuni e gestori degli impianti \\
\midrule
\textbf{{\color{Plum} Iscrizione a IRRIFRAME}} & Servizio che permette alle aziende registrate di salvare il proprio profilo colturale e di personalizzare l'informazione irrigua fornita dal servizio comunicando in tempo reale dati locali \\
\midrule
\textbf{{\color{Plum} Link utili}} & Una rassegna di 'siti utili'; la suddivisione per argomenti e temi permette di individuare facilmente i riferimenti cercati \\
\midrule
\textbf{{\color{Plum} Richiesta pubblicazioni}} & Possibilità di richiedere una copia delle pubblicazioni edite da ARPAV attraverso posta elettronica e la compilazione di un modulo \\
\bottomrule

\caption{Servizi Online di ARPAV}
\end{longtable}

\section{Organizzazione Interna}

ARPAV è un'agenzia regionale che si lavora per il completamento di progetti a livello regionale, nazionale ed internazionale ed offre tutta una serie di servizi a livello regionale e non, tutto riguardante l'ambito ambientale. Per fare ciò necessita di una strutturata organizzazione interna, suddivisa in vari organi operativi, ognuno dei quali ricopre una funzionalità specifica e si interfaccia con un altra secondo protocolli prestabiliti. ARPAV è dotata di una autonomia interna negli ambiti amministrativi, organizzativi e tecnico contabile.

\subsection{Processi di Sviluppo}

[:] Descrizione dei processi attuati all'interno dell'agenzia per lo sviluppo dei prodotti e la fruizione dei servizi e i metodi utilizzati per i processi di sviluppo.

\subsection{Metodologie di Supporto ai Processi}

[:] Elenco e descrizione delle metodologie utilizzate a supporto dei processi dall'agenzia.

\subsection{Strumenti di Supporto ai Processi}

[:] Elenco e descrizione degli strumenti utilizzati dall'agenzia a supporto dei processi.

\subsection{Tecnologie Utilizzate}

[:] Elenco e descrizione delle tecnologie utilizzate dall'agenzia in relazione all'ambito in cui vengono applicate.

\section{Relazioni Esterne}

[:] Descrizione dei clienti diretti dell'agenzia e la metodologia di interazione; breve descrizione dei clienti secondari che usufruiscono dei prodotti erogati dall'agenzia.

\subsection{Clienti Diretti}

[:]  descrizione particolare dei clienti diretti dei servizi e prodotti dell'agenzia e le metodologie di relazione con essi

\subsection{Clienti Secondari}

[:] descrizione dei clienti secondari, ma spesso finali dei prodotti e servizi forniti dall'agenzia.

\subsection{Orientamento all'Innovazione}

[:] Descrizione della propensione dell'ente regionale all'innovazione in relazione con i suoi prodotti e servizi e progetti futuri.

