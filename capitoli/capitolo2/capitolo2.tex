\chapter{Lo Stage per ARPAV}
\label{2.0}
\thispagestyle{fancy} 

Il dipartimento di reti ed informatica deve affrontare la difficile sfida di mantenere la parità del bilancio, senza rinunciare alla qualità delle loro opere. Il responsabile del dipartimento, spinto da questa necessità, ha orientato la politica di scelta del personale, durante la fase di codifica, verso stagisti neo-laureati o laureandi. Questa scelta permette:

\begin{itemize}

	\item \textbf{maggior efficacia:} l'utilizzo di stagisti permette di ottimizzare le risorse di bilancio, diminuendo l'impatto sul \textit{budget};
	\item \textbf{maggiore efficacia:} menti fresche di studio e con bassa specializzazione permettono un'adesione a progetti nuovi ed innovativi con maggiore facilità rispetto a personale propenso a lavorare secondo processi prestabiliti.

\end{itemize}

[:] introduzione del capitolo con la descrizione dei motivi per cui l'agenzia ricerca stagisti per affidare loro dei progetti da sviluppare.

\section{L'Esigenza di uno Stage}

Il dipartimento di reti ed informatica, al momento del mio inserimento, stava seguendo la progettazione e lo sviluppo di un pluviometro\ped{g} innovativo a basso costo da parte del \textit{FabLab}\ped{g} di Verona. Questo incarico è stato proposto da parte di ARPAV per il progetto RE.S.M.I.A., il quale prevede il potenziamento dell'infrastruttura della rete di monitoraggio di ARPAV, tramite la progettazione di nuove stazioni meteorologiche.
Contemporaneamente al progetto \textit{FabLab}, il responsabile del dipartimento stava seguendo il \textit{Gruppo Meteo} di \textit{RaspiBO}\ped{g} il quale sta sviluppando stazioni meteorologiche a basso costo, al momento sprovviste di pluviometro. Così si è riscontrata la necessità che qualcuno progettasse e sviluppasse la parte software per la gestione del pluviometro sperimentale e le connessioni \textit{hardware} per integrare parte mancante del progetto volto all'innovazione delle apparecchiature di monitoraggio ambientale.


[:] Sezione in cui descrivo i motivi per cui l'agenzia ha trovato la necessità di avviare questo specifico stage per questo progetto

\section{Presentazione del Progetto}

[:] Sezione in cui descrivo il progetto vero e proprio, in che cosa consiste e in quali realtà potrà essere inserito.

\subsection{Obbiettivo dello Stage}

[:] Sottosezione in descrivo gli obbiettivi che sono stati preposti all'inizio dello stage nel momento in cui ho preso visione dell'offerta fatta dall'agenzia.
La sezione inizia con una breve introduzione del piano di lavoro e la descrizione del significato di obbiettivo.

\subsubsection{Obbiettivi Minimi Richiesti}

[:] Elenco e descrizione degli obbiettivi minimi richiesti all'inizio dello stage.

\subsubsection{Obbiettivi Massimi Raggiungibili}

[:] Elenco e descrizione degli obbiettivi che l'agenzia desidera raggiungere, ma che non sono vincolanti alla conclusione dello stage.

\subsection{Finalità del Progetto}

[:] Sezione descrivo in dettaglio l'applicazione del prodotto dello stage nella realtà dell'ente e non.

\section{Vincoli di Progetto}

[:] Descrizione delle tipologie di vincoli che sono stati imposti nello stage.

\subsection{Vincoli di Dominio}

[:] Descrizione dei vincoli che dovranno essere soddisfatti dal progetto in termini di funzionalità, affidabilità e manutenibilità. 

\subsection{Vincoli Tecnologici}

[:] Descrizione dei vincoli riguardanti gli strumenti di lavoro, come framework e  e tecnologie obbligatori che dovranno essere utilizzati durante l'attività stagistica.

\subsection{Vincoli Metodologici}

[:] Descrizione dei vincoli metodologici che lo ho dovuto seguire durante tutta la fase dello stage.

\subsection{Vincoli Temporali}

[:] Descrizione dei vincoli temporali prefissati in base alle necessità dell'ente e i problemi di fattori esterni dovuti alla cooperazione di aziende terze per il completamento dello stage.
